\chapter{函数}
我们已经学习了以基本和高级数据类型为代表的算子、运算符、由算子运算符组合而成的表达式、语句、以及控制语句流向的控制结构。本章学习如何将所有这些内容组织成函数。函数可以一次编写,反复使用,避免重复写代码。重复的代码,意味着,重复劳动,而且修改一处,必得处处修改,保持一致,非常不利于维护。对程序员而言,其职业特质之一就是:懒,意指站在巨人肩膀上,不做无谓的重复工作。采用的主要手段叫做代码重用(reusable)
\section{从问题抽象出函数}
实现代码重用的方式之一是抽象。抽象是使人能理解计算机程序的关键步骤。对计算机而言, 再多的代码都无所谓,越详细越好。人却不行,人的思维方式必须用抽象,分治的方式降低事物复杂度,才好理解。设计程序也应该人的思维方式,抽象一点。

比如,设计程序完成如下任务:下载网页,统计网页中的词频,输出词与对应的频率。这理解起来很容易,转成python如下:
\begin{python}
  # psuedo code
  page = download_page()
  freqs = compute_frequencies(page)
  for word, freq in freqs:
      print(word, freq)
\end{python}
抽象的意义在于,向使用者隐藏所有实现细节,仅提供函数签名作为接口(或称协议)。Python中协议,指控制行为的规则。设计程序时,首先告诉计算机做什么,而暂不考虑如何做。如何做要放到函数定义中。另一方面,能观察到,伪码与Python语言的转换十分自然流畅。
\section{定义和调用}
函数是可以调用的,调用后,函数执行某些语句,返回值。在Python中可以调用的事物称为可调用的(callable)。函数是典型的callable。Python有七个可调用类型
\begin{itemize}
\item 用户定义的函数 user-defined functions
\item 内置函数 built-in functions
\item 内置方法 built-in methods
\item 类方法 methods
\item 类 classes
\item 类实例 Class instances 
\item 生成器函数 Generator functions
\end{itemize}
下面例子演示如何判断一个对象是否是可调用的。
\begin{python}
import math
x  = 1
y = math.sqrt  # note no
x.__call__ # error
y.__call__
\end{python}
结构化程序设计,最核心的事情就是创建函数和使用函数。 一般,先写代码结构,再填空。以至于有人说,写程序不难,就是填空。看下面这个例子。
\begin{framed}
\begin{verbatim}
def 函数名(参数列表):
    """
    docstring: 函数说明,说明函数功能,使用何种算法,参数个数、类型,返回值等。
    """
    #  注释

    代码块(函数体)
    1. 输入 input
    2. 处理
    3. 输出

    return 返回值
\end{verbatim}
\end{framed}
我们来填空。
\begin{framed}
\begin{verbatim}
# 函数的定义
def fibs(num):
    """Calculate fibonacci nubmers

  num: length of the output
    """

    # 1.初始化及输入
    result = [0, 1]

    # 2.处理
    for i in range(num - 2):
        result.append(result[-2] + result[-1])  # 3.输出

  # 返回值/输出
    return result

# 函数的调用
fibs(10)
fibs(15)
\end{verbatim}
\end{framed}
写docstring背后的理念叫做:程序文档化,即程序是自解释的self-explaining。试下面两个命令。
\begin{python}
fibs.__doc__
help(fibs)
\end{python}
\subsection{Return Value}
函数表面看,可以返回值,也可以不返回值,可写return带值,也可以不写return语句。
\begin{python}
def test():
    print('This is printed')
    return
    print('This is not')

x = test()
print(x)  # None
\end{python}
实际上所有函数都返回值,当什么都不返回时,实际返回的是None值。
\subsection{Py File Extensions}
函数保存到py文件中。
\begin{itemize}
\item py: python源文件,文本文件,由解释器解释执行
\item pyc: python字节码文件,二进制文件,用于隐藏python源代码和提高运行速度。python模块第一次被引入时将被编译成字节码文件,以后再次导入时,优先使用.pyc文件,提高加载和运行模块速度
\item pyw: python源文件,常用于图形界面程序
\end{itemize}
\section{Arguments}
定义函数时圆括号内用逗号分割形式参数(formal parameters)列表。调用函数时,调用者向函数传递实际参数(actual parameters or arguments)。函数通过参数得到外界值。
\begin{python}
def try_to_change(n):
    n = 'Mr.Gumby'

name = 'Mrs. Entity'
try_to_change(name)
name
\end{python}
同名变量,函数内局部变量屏蔽(shadowing)外部的局部变量或全局变量。上例的实际参数是字符串,它是不可改变类型的,如果以可改变类型数据做参数会怎样呢?看下面例子,试解释程序执行结果。
\begin{python}
def change(m):
    m[0] = 'Mr.Gumby'

mlist = ['Mrs. Entity', 'Mrs. Thing']

change(mlist)

mlist
\end{python}
前例是将值赋予局部变量,所谓“传值调用(call by value)”。本例是将列表名字(name),传给变量m,因此m和mlist两个名字,指向的是同一个值(列表),修改m指向的列表时,mlist自然改变,所谓“传引用调用(call by reference)”。
\paragraph{思考: }  如何将列表传递改为传值调用?

\section{Argument Type}
\subsection{Positional arguments}
位置参数 (positional parameters),位置起重要作用,比参数名字更重要。
\begin{python}
def hello1(greeting, name):
    print('%s, %s!' % (greeting, name))

def hello2(name, greeting):
    print('%s, %s!' % (name, greeting))

hello1('hello', 'world!')
hello2('HELLO', 'WORLD!')
\end{python}
可看出参数名字不甚重要,尽管参数名字不同,这两个函数做的是同一件事。
\subsection{Keyword Argument}
当参数过多时,不容易记住参数的顺序。为了简化记忆,可以给参数起名字。关键字参数(keyword parameters),能用名字明确示意参数含义和功能。用关键字参数时,顺序不重要了,名字重要。
\begin{python}
  hello1(greeting='hEllo', name='world')
  hello1(name='world',greeting='hEllo')
  hello2(greeting='heLLo', name='world')
\end{python}
使用关键字参数的函数,优势在于清楚表明,该函数需要什么样的参数,有助于理解函数要表达的意思的。比如调用函数\pyth{store('Mr. Brainsample', 10, 20, 13, 5)},很难一眼看出这是在做什么。如果写法改成\pyth{store(patient='Mr. Brainsample',hour=10,minute=20, day=13, month=5)},几乎可以猜到该函数完成什么功能了。关键字参数另一个让人眼前一亮之处,是在函数定义时,可定义参数默认值。再也不用为少传递一个参数引起报错担心了。

位置参数和关键字参数可以组合使用,使用时要求位置参数在前,关键字参数在后。
\begin{python}
def hello4(name='world', greeting='Hello'):
    print('%s, %s!' % (name, greeting))

hello4()
hello4('Greetings')
hello4('Greetings', 'universe')

def hello5(name, greeting='Hello', punctuation='!'):
    print('%s, %s%s' % (greeting, name, punctuation))

hello5('Mars')
hello5('Mars', 'Howdy')
hello5('Mars', 'Howdy', '...')
hello5('Mars', punctuation='.')
hello5('Mars', greeting='Top of the morning to ya')
hello5()
\end{python}

注意位置参数和关键字参数顺序。
% \subsection{可变长参数}
\subsection{Variable Number Of Arguments}
可变长参数,指有时无法在定义函数时确定参数个数,因此函数必须允许用户提供任意多的参数。比如,\pyth{store(data, name1, name2, name3)},此种情况应该允许提供任意多的名字。为此,要在函数定义中使用可变长参数。语法是\pyth{function(*args), function(**kwargs)}
\begin{python}
def print_params(*params):
    print(params)
    print_params('Testing') print_params(1, 2, 3) # the argument is a
    tuple

def print_params_2(title, *params):
    print(title)
    print(params)

print_params_2('Params:', 1, 2, 3)
print_params_2('Nothing:')  # get an empty tuple
\end{python}

一个星号表示把余下所有的位置参数集到一个元组里,该符号只能收集位置参数,却无法收集关键字参数。
\begin{python}
print_params_2('Hmm...', something=42)  # TypeError
\end{python}
需要使用另外一个操作符,收集所有关键字参数。两个星号将关键字参数收集到字典里。
\begin{python}
def print_params_3(**params):
    print(params)

print_params_3(x=1, y=2, z=3)

def print_params_4(x, y, z=3, *pospar, **keypar):
    print(x, y, z)
    print(pospar)
    print(keypar)

print_params_4(1, 2, 3, 4, 5, 6, 7, foo=1, bar=2)  # can you explain the result?
\end{python}
% \subsection{反向过程}
\subsection{Reverse Processing}
可变长的作用是收集参数,其反过程是将收集的参数打散分配给变量。在调用时,给参数前加星号,即可将参数分配给变量。一个星号适用于元组、列表,两个星号,适用于字典。定义函数和调用函数,仅一方使用时,才能体现可变参数优势。都使用或都不使用,体现不出优势。
\begin{python}
def add(x, y):
    return x + y

params = (1, 2)
add(*params)  #  for tuple, list

params = {'name': 'Sir Robin', 'greeting': 'Well met'}
hello2(**params)  # for dictionary

# onle used in one side is good
def with_stars(**kwds):
    print(kwds['name'], 'is', kwd['age'], 'years old')

def without_stars(kwds):
    print(kwds['name'], 'is', kwds['age'], 'years old')

args = {'name': 'Mr. Adam', 'age': 23}
with_stars(**args)
without_starts(args)
\end{python}
% \subsection{练习阅读理解下面代码}
\subsection{Exercise: Code Reading}
\begin{python}
def story(**kwds):
    return 'Once upon a time, there was a ' \
    '%(job)s called %(name)s.' % kwds

def power(x, y, *others):
    if others:
        print('Received redundant parameters:', others)
    return pow(x,y)

def interval(start, stop=None, step=1):
    'Imitates range() for step > 0'
    if stop is None:
        start, stop = 0, start
    result = []
    i = start
    while i < stop:
        result.append(i)
        i += step
    return result

print(story(job='king', name='Gumby'))
print(story(name='Sir Robin', job='brave knight'))
params = {'job': 'language', 'name':'Python'}
print(story(**params))

del params['job']
params
print(story(job='stroke of genius', **params))


power(2, 3), power(3, 2)
params = (5,) * 2
print(params)
power(*params)

power(3,3, 'Hello, World')

interval(10), interval(1,5), interval(3, 12, 4)

power(*interval(3,7))
\end{python}

\section{Scope, Namespace}
变量是什么? 可以吧变量理解为指向值的名字、标签。\pyth{x = 1}就好像字典项一样,只不过是一个“无形”字典。内置函数vars,其返回值,就是这个描述变量的“无形”字典。
\begin{python}
x = 1
scope = vars()
scope['x']
scope['x'] += 1
x
\end{python}
这种无形的字典称为命名空间(namespace)或作用域(scope)。全局空间是一个命名空间,每一个函数调用都会创建一个新的命名空间。所以,函数的局部变量仅在其局部命名空间中起作用。传值参数是典型的局部变量。
\begin{python}
def output(x):
    x = 42
    print(x)

x = 1
y = 2
output(y)
\end{python}
如果要在函数内访问全局变量怎么办?
\begin{python}
  def combine(param):
      print(param + external)

  external = 'berry'
  combine('Shrub')
\end{python}

可以在函数内重新绑定(rebinding)变量。默认函数内变量都是局部变量,通过重新绑定,设为全局变量。
\begin{python}
x = 1
def change_g():
    global x  
    x = x + 1

change_g()
x
\end{python}
写程序,应尽量减少使用全局变量,尽量不要在函数内绑定全局变量。 频繁使用全局变量是不良编程风格。

% \subsection{递归}
\subsection{Recursion}
函数可以调用其他函数。而且,函数还可以调用自身(函数),调用自身称为递归。Recursion: referring to yourself。

递归关键是识别出两种情况:
\begin{itemize}
\item 基础情况 A base case 最简单情况,函数直接返回值
\item 递归情况 A recursive case 调用自身,但规模问题规模减小了
\end{itemize}
写递归程序步骤:
\begin{enumerate}
\item 写出基础情况和递归情况
\item 阅读代码
\item 解释代码
\item 修改代码
\end{enumerate}
\subsubsection{练习:阶乘和乘方}
\begin{python}
def frac(n):
    result = n
    for i in range(1, n):
        result *= i
    return result

# + base case: the fractorial of 1 is 1
# + recursive case: the fractorial of n greater than 1 is the product of
#  n and the fractorial of n-1

def frac(n):
    if n == 1:
        return 1
    else:
        return n * frac(n-1)
    print('n in this level is %d' % n)

def power(x, n):
    result = 1
    for i in range(n):
        result *= x
    return result

# + power(x, 0) is 1 for all number x.
# + power(x, n) for n > 0 is the product of x and the power(x, n-1)

def power(x, n):
    if n == 0:
        return 1
    else:
        return x * power(x, n-1)
\end{python}

\subsubsection{练习: 二分法}
\begin{python}
def search(sequence, number, lower=0, upper=None):
    if upper is None:
        upper = len(sequence) - 1
    if lower == upper:
        assert number == sequence[upper]
        return upper
    else:
        middle = (lower + upper) // 2
        if number > sequence[middle]:
            return search(sequence, number, middle+1, upper)
        else:
            return search(sequence, number, lower, middle)

seq = [34, 67, 8, 123, 4, 100, 95]
seq.sort()
seq
search(seq, 34)
\end{python}

\section{Batteries Included}
% \subsection{模块中的函数}
\subsection{Modules}
Python真正强大、吸引人之处在于其内置及海量的第三方工具包,\href{https://pypi.python.org/pypi}{PYthon Package Index}。安装Python时自带的内置工具包称为标准库。第三发开发的称为第三方库或扩展库。任意python程序都能被引入为模块。
\begin{python}
# hello2.py
def hello():
    print("Hello world!")

import hello2
hello2.hello()  
\end{python}
模块被引入时,主要工作是定义变量、函数、类等。定义只要一次就够了,多次引用一个模块,与引用一次效果相当。引入模块后,该模块会保持其作用域,该模块中定义的变量、函数、类自动成为该模块属性。
\begin{python}
import constants
print(constants.PI)
\end{python}
模块既可以独立运行,也可以被引入。
\begin{python}
# hello3.py
def hello():
    print("hello, world!")

# a test:
hello()

# The key to avoiding it is "telling" the module whether it's being run
# as a program on its own or being imported into another program.  To do
# that,  you need the variable __name__:

# hello4.py
def hello():
    print("hello world!")

def test():
    hello()

if __name__ == '__main__':
    test()
\end{python}
\subsection{Module Path}
第三方模块一般安装在site-packages文件夹中。
\begin{python}
import sys, pprint
pprint.pprint(sys.path)
\end{python}
% \subsection{包}
\subsection{Package}
当模块多起来时,需要组织成包(package)。包是另一种类型的模块,包可以包含其他模块。可以将包视为文件夹。一个包能从文件夹变身为Python能识别的包,它必须包含一个名为\pyth{__init__.py}的文件,该文件描述包的内容。

% \section{程序打包}
\section{Packaging}
%\subsection{创建程序包}
\subsection{Create Files and Path}
创建目录结构如下
\footnote{\url{https://packaging.python.org/tutorials/packaging-projects/}}
\begin{framed}
\begin{verbatim}
/packaging_tutorial
  /example_pkg
    __init__.py
\end{verbatim}
\end{framed}
\pyth{__init__.py}告诉Python,\pyth{example_pkg}不仅是个文件夹,而且是一个Python包。编辑\pyth{__init__.py},写入\pyth{name = "example_pkg"}。

一个包常包含几个特定文件。
\begin{framed}
\begin{verbatim}
/packaging_tutorial
  /example_pkg
    __init__.py
  setup.py
  LICENSE
  README.md
\end{verbatim}
\end{framed}
\pyth{setup.py}提供了包的信息,如名字、版本、源文件等,打包工具\pyth{setuptools}根据该文件内的信息进行打包。读者打包时,须用自己包的信息替换模板中的信息。
\begin{python}
import setuptools

with open("README.md", "r") as fh:
    long_description = fh.read()

setuptools.setup(
    name="example-pkg-your-username", # ffrreedd
    version="0.0.1",
    author="Example Author", # Fred
    author_email="author@example.com",
    description="A small example package",
    long_description=long_description,
    long_description_content_type="text/markdown",
    url="https://github.com/pypa/sampleproject",
    packages=setuptools.find_packages(),
    classifiers=[
        "Programming Language :: Python :: 3",
        "License :: OSI Approved :: MIT License",
        "Operating System :: OS Independent",
    ],
)
\end{python}
\pyth{README.md}是用Markdown格式写的使用说明。
\begin{framed}
\begin{verbatim}
# Example Package

This is a simple example package. You can use
Github-flavored Markdown:
https://guides.github.com/features/mastering-markdown/
to write your content.
\end{verbatim}
\end{framed}
\pyth{LICENSE}是版权文件。参看:\href{https://choosealicense.com/}{如何选择版权}。

\subsubsection{创建Python包}
打包需要使用\pyth{setuptools, wheel}两个工具,确保已安装这两个工具包。没安装,用如下命令安装。
\begin{framed}
  python3 -m pip install --user --upgrade setuptools wheel
\end{framed}

在\pyth{setup.py}所在目录下,执行下面命令。
\begin{framed}
\begin{verbatim}
  python3 setup.py sdist bdist_wheel
\end{verbatim}
\end{framed}

该命令生成一个名为dist的文件夹,内有两个文件。
\begin{framed}
\begin{verbatim}
dist/
  example_pkg_your_username-0.0.1-py3-none-any.whl
  example_pkg_your_username-0.0.1.tar.gz
\end{verbatim}
\end{framed}
\pyth{*.tar.gz}是打包压缩的源代码,\pyth{whl}是打包后的二进制文件,即常说的“发行版”。

可以将该发行版拷给同学使用,也可以将该发行版上传到Pypi供更多人使用。wheel是Python文件的二进制格式。安装wheel文件命令:
\begin{framed}
\begin{verbatim}
  python install example_pkg_your_username-0.0.1-py3-none-any.whl
\end{verbatim}
\end{framed}
使用下面命令,看该包安装到哪里去了。
\begin{framed}
\begin{verbatim}
    pip list
\end{verbatim}
\end{framed}
如果安装好了,则可以使用该包了。使用下面命令,看看结果是什么。
\begin{framed}
\begin{verbatim}
import example_pkg
example_pkg.name  # 'example_pkg'
\end{verbatim}
\end{framed}
\subsubsection{练习:打包}
写一个程序,打包,将程序的whl文件,发布给同学使用。看看你写的程序能否为同学们使用。然后,听听同学们的意见,根据反馈,修改程序,发布第二个版本。
使用github练习多人协作编写较大的程序。
\section{File}
为长期保存数据,须将数据以文件形式存储到磁盘中。文件分为文本文件、二进制文件。
文件操作流程如下。
\begin{enumerate}
\item 打开文件并创建文件对象
\item 通过文件对象对文件内容进行读取、写入、修改、删除
\item 保存文件,关闭文件对象
\end{enumerate}
\subsection{Open}
\pyth{open(name[, mode[, buffering]])}函数打开文件。扩展巴克斯范式Extended Backus-Naur form用于描述函数。其中,name是必要(madatory)参数,mode和buffering是可选(optional)参数,该函数返回一个文件对象。

\begin{python}
f = open(r'c:\text\somefile.txt')
\end{python}
自此,f文件对象代表该文件。

\subsection{Open Mode}
\begin{table}
\centering
\begin{tabular}{cc}
  \toprule
  值  & 说明 \\
  \midrule
  'r' & 只读 \\
  'w' & 写入 \\
  'a' & 追加 \\
  'b' & 二进制模式(可组合) \\
  '+' & 读写模式(可组合) \\
  \bottomrule
\end{tabular}
\caption{文件打开模式}
\end{table}
一般,python默认处理文本文件,如处理二进制文件需要明确指定二进制模式。缓存标志位,0表示不缓存,1表示使用缓存。
\subsection{File Operation}
\subsubsection{读写文件内容}
上下文(context)管理关键字with自动管理资源,不论何种原因跳出with块,总能保证文件正确关闭。读写文件时,用这种方式打开文件。
\begin{python}
f = open('somefile.txt', 'w')
f.write('1')
f.write('2')
f.close()

f. = open('somefile.txt', 'r')
f.read(4)
f.read()

# recommend syntax, using context
with open('somefile.txt', 'w') as f:
    f.write('Hello,')
    f.write('World!')
    
\end{python}

\subsubsection{按行读写文件}

\begin{python}
with open('sample.txt','r') as f:
    while True:
        line = f.readline()
        if line == '':
            breal
        print(line, end='')


# sample.txt
# Welcome to this file
# There is nothing here except
# this strange line

f = open(r'c:\text\sample.txt')
f.read(7)
f.read(4)
f.close()

f = open(r'c:\text\sample.txt')
print(f.read())
f.close()

f = open(r'c:\text\sample.txt')
for i range(3):
    print(str(i) + ":" + f.readline())
f.close()

import pprint
pprint.pprint(open(r'c:\text\sample.txt').readlines()) # file object is being closed automatically.

# this
# is no
# haiku
lines[1] = "isn't a\n"
f = oepn(r'c:\text\sample.txt', 'w')
f.writelines(lines)
f.close()

# one char a time
def process(string):
    print('Processing: ', string)

f = open(filename)
while True:
   char = f.read(1)
   if not char: 
        break
   process(char)
f.close()

# one line a time
f = open(filename)
while True:
   line = f.readline()
   if not line: 
        break
   process(line)
f.close()

# 
f = open(filename)
for char in f.read():
    process(char)
f.close()

f = open(filename)
for line in f.readlines():
    process(line)
f.close()

# Lazy Line Iteration with fileinput
# lazy line iteration
import fileinput
for line in fileinput.input(filename):
    process(line)

# file objects are iterable
f = open(filename)
for line in f:
    process(line)
f.close

# Iterating over a file without storing the file object in a variable
for line in open(filename):
    process(line)

\end{python}


\subsection{读写二进制文件、操作目录}
\subsubsection{读写二进制文件}
Python使用pickle模块,操作二进制文件。pickle是常用且速度快的二进制文件序列化模块。使用pickle进行对象序列化和二进制文件读写。

\begin{verbatim}
import pickle

dir(pickle)  #  查看pickle模块属性

#  定义数据
f = open('sample_pickle.dat', 'wb')
n = 7
i = 13000000
a = 99.056
s = '中华人民共和国 abc 123'
lst = [[1, 2, 3], [40, 50, 60], [7, 8, 9]]
tpl = (-5, 10, 8)
clctn = {4, 5, 6}
dct = {'a': 'apple', 'b': 'banana', 'g': 'grape', 'o': 'orange'}

#  将数据写入二进制文件
try:
    pickle.dump(n, f)  #  转储
    pickle.dump(i, f)
    pickle.dump(a, f)
    pickle.dump(s, f)
    pickle.dump(lst, f)
    pickle.dump(tpl, f)
    pickle.dump(clctn, f)
    pickle.dump(dct, f)
except:
    print('Writing file exception')
finally:
    f.close()

#  从二进制文件读入数据,并输出到屏幕
try:
    f2 = open('sample_pickle.dat', 'rb')
    n = pickle.load(f2)
    i = 0
    while i<n:
        x = pickle.load(f2)
        print(x)
        i += 1
except Exception as e:
    print(e)
finally:
    f2.close()
\end{verbatim}
\subsubsection{操作目录}
处理文件路径,要使用\pyth{os.path}模块中的对象和方法。该模块提供大量用于路径判断、切分、连接及文件夹遍历方法

例: 查找特定后缀文件
\begin{python}
import os

# dir(os.path)

print([fname for fname in os.listdir(os.getcwd()) if os.path.isfile(fname) and fname.endswith('.org')])
\end{python}
例: 遍历文件夹
\begin{python}
import os

for root, dirs, files in os.walk(".", topdown=False):
    for name in files:
        #  print(os.path.join(root, name))
        pass
    for name in dirs:
        #  print(os.path.join(root, name))
        pass
\end{python}

\subsubsection{练习}
简单爬取网页、解析文本程序

\begin{python}
from urllib.request import urlopen
from BeautifulSoup import BeautifulSoup  # install BeautifulSoup

html = urlopen('http://python.org/community/jobs').read()
# print(html)
soup = BeautifulSoup(text)

jobs = set()
for header in soup('h3'):
    links = header('a', 'reference')
    if not links: 
        continue
    link = links[0]
    jobs.add('%s (%s)' % (link.string, link['href']))

print('\n'.join(sorted(jobs, key=lambda s: s.lower())))
\end{python}

\section{如何自学}
当遇到一个模块,自己应尽力了解、探索,对模块有个大致掌握。为后续深入钻研打个基础。

以copy为例,为了了解copy梗概,我们在解释器提示符下,探索该模块。
\subsection{dir}

\begin{python}
import copy
dir(copy)  
\end{python}
\pyth{dir()}列出模块中的属性、函数、类、变量等。有些名字带下划线,下划线是作者按python编程惯例明示,这些名字不应该在模块外使用,即这些名字代表的函数或类是作者自用的,而不是给用户用的。
\begin{python}
  [ n for n in dir(copy) if not n.startswith('_')]
\end{python}
看看上面这个程序做了什么。
\subsection{\pyth{__all__}}
\pyth{copy.__all__}会得到一个列表,列出可用函数。与上一个列表推导式基本一致。区别在于,\pyth{__all__}的内容是由作者拟定的。

看copy.py里面的\pyth{__all__}变量声明,一切可知。

当使用\pyth{from copy import *}得到的正是\pyth{__all__}中列出的函数。当自己写模块时,也应在模块中声明\pyth{__all__}变量。明确告诉读者,哪些函数是可用的,是个好习惯。
\subsection{help}
\pyth{help(copy.copy)}给出的信息,基本够一个程序员正常所需。
\subsection{\pyth{__doc__}}
\pyth{print(copy.copy.__doc__)}显示copy模块中copy函数的\pyth{__doc__}属性\pyth{help}一般比doc给出更多关于模块的信息。
\subsection{Library Reference}
\href{https://docs.python.org/3.5/library/index.html}{Python Library Reference(for Python 3.5)}描述了所有内置库的模块。遇到问题,尤其是想了解关于Python的具体情况时,基本会在这里找到答案。
\subsection{\pyth{__file__}}
真想理解Python,真想透彻搞清一个模块,除了阅读源代码,没有更好的办法了。读Python源码是第二有效的学习Python的方法。第一有效呢?动手写Python代码。那么,如何找到源码文件?先要找到该源文件的位置,\pyth{print(copy.__file__)}。到这个文件夹下,打开文件,开始阅读理解这个模块的工作原理和技术细节吧。

\section{练习:自学以下内置模块}
\begin{python}
  sys, os, file, input, sets, heapq, collections, time, random,
  shelve, re, functools, difflib, hashlib, csv, timeit, profile,
  trace, datetime, itertools, logging, getopt, optparse, cmd, re
\end{python}

\pyth{re}实现“正则表达式(regular expression)”,用于匹配字符串。关于正
则表达式,参考\footnote{余晟.正则指引.电子工业出版社,2012},
\footnote{Jeffrey E.F.Friedl著,余晟译.精通正则表达式.电子工业出版社,2012}。


\section{练习例子}
\pyth{reExample.py} \\
\pyth{Mudball.py} \\
\pyth{howdoi}

